\documentclass[a4paper,12pt]{article}

%\pagestyle{empty}
\usepackage{geometry}
\usepackage[ngerman]{babel}
\usepackage[utf8x]{inputenc}
\usepackage[T1]{fontenc}
\usepackage[pdftex]{hyperref}
\usepackage{amsmath,amssymb,listings,tikz}

%more space
\geometry{top=2cm, right=2cm, bottom=2cm, left=2cm}

%no numbers
\def\thesection{}
\def\thesubsection{}
\def\thesubsubsection{}

\usetikzlibrary{positioning,automata}

\title{Übungsblatt 1}
%\author{Alexander Vogelgsang\\Alexander Diefenbach}

\begin{document}
\begin{minipage}[t]{0.7\textwidth}
\begin{flushleft}
Universität Augsburg, Institut für Informatik\\
Prof. Dr. Bernhard Möller\\
%Christoph Etzel, Simon Lohmüller, Thomas Driessen
\end{flushleft}
\end{minipage}
\begin{minipage}[t]{0.3\textwidth}
\begin{flushright}
Sommersemester 2016\\
Vorlesung Graphikprogrammierung\\
Abgabe \today
\end{flushright}
\end{minipage}
\begin{center}
\vskip 3em
{\huge \bfseries Übung zu Graphikprogrammierung\\[1em]}

\makeatletter
{\large \@title\\[1em]}
\makeatletter
\@author
\makeatother
\end{center}




\section{Aufgabe 1}
$A=(56,34,23), B=(43,88,15), C=(33,77,44)$\\
$\overrightarrow{AB} = \begin{pmatrix}-13\\54\\-8\end{pmatrix}
\overrightarrow{BC} = \begin{pmatrix}-10\\-11\\29\end{pmatrix}
\overrightarrow{AB} \times \overrightarrow{BC} =\begin{pmatrix}1478\\457\\683\end{pmatrix} \not= \begin{pmatrix}0\\0\\0\end{pmatrix} \Rightarrow \text{Punkte liegen nicht auf einer Geraden}$
\section{Aufgabe 2}
$ A=(4,-5,-2), B=(12,-9,2), C=(1,-4,5)$\\
$\overrightarrow{AB} = \begin{pmatrix}8\\-4\\4\end{pmatrix}
\overrightarrow{AC} = \begin{pmatrix}-3\\1\\7\end{pmatrix}$
$\overrightarrow{AB} \cdot \overrightarrow{AC} = 0 \Rightarrow \text{Dreieck ist rechtwinklig}$
\section{Aufgabe 3}
$A=(-3,4,5), X=(1,0,0), Y=(0,1,0), Z=(0,0,1)$\\
$\alpha = \cos^{-1}(\frac{\vec{a} \cdot \vec{b}}{||\vec{a}|| \cdot ||\vec{b}||})$\\
$\alpha_x = 2.00895 = 115.104^\circ \quad
\alpha_y = 0.969532 = 55.5501^\circ \quad
\alpha_z = 0.785398 = 45^\circ$
\end{document}